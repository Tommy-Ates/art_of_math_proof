\documentclass[a4paper,12pt]{article}
\usepackage{amsfonts}
\begin{document}

\title{The Art of Mathematical Proof}
\author{Will Dengler}
\maketitle

\section{Introduction}
What is a math proof? That is after all why you are reading this article. Well I'm not going to answer that, at least not yet. First, I want you to think about the following statements:
\begin{itemize}
	\item The sum of two even integers is even.
	\item The sum of two odd integers is even.
	\item The sum of an even and an odd integer is odd.
\end{itemize}

You know these statements to be true, but how do you know they are true? Maybe you quickly added some numbers in your head to confirm that each statement is true for every example you ran through. That may be good enough to have satiated any doubt you had, but it doesn't answer the more important question - why are these statements true? All of mathematical theory is focused around answering the question of why something is true; and the tool used is the mathematical proof. 

You should now have some understanding of what a mathematical proof is - it's the tool used to answer the question of why something is true; put bluntly, a mathematical proof is the proof (pun intended) that a statement is true. Okay great, so what exactly does a math proof look like? From a high-level, a math proof is a well-structured, logical argument showing that a statement is true. How does it do this? A proof starts with the facts at hand "I have two even integers, therefore I can represent the first number as  $2*j$ and the second as $2*k$, where $j$ and $k$ are integers because even numbers are defined to be $2$ times some integer.", then deduces what must be true given those facts,"Therefor, I can represent their sum as $(2*j + 2*k)$". With the new information, more facts can then be uncovered "We can then factor out the $2$ from this representation:  $(2*j + 2*k) = 2*(j + k)$. The number $(j + k)$ must be an integer because the sum of two integers is an integer." This process continues on until the truth of the statement is uncovered, "I've shown that the sum of the two even numbers is equal to $2$ times the integer $(j + k)$, which means that this number is even by definition.". 

Don't worry if you don't see why the argument above does in fact prove that the sum of two integers is even, we'll come back to it soon. Instead, focus on the nature of the argument itself. You may have noticed that there's no references to specific examples; that each 'step' in the argument was coupled with a justification; or the reference to a specific definition of what it means to be even - these are the hallmarks of a proof. 

Okay, that's great Will, can you now tell me exactly what a proof is, and how I can go about writing one? Well I can't... there's no easy way to define what a proof is, nor a simple set of rules I can give for writing one - it'd be like defining the color red, sure I could tell you its wavelength, but that doesn't come close to explaining it. What I can do however is teach you how to recognize what is a proof and what's not, as well as the tools to begin to start proving things on your own.

\section{The Fundamentals of Theory}
In mathematics, definitions are the building blocks of all theory. A definition unambiguously and efficiently describes an 'object'. From the definition springs all math theory - first, the properties that are inherent to the definition of the object are proved. From there, more complex facts are uncovered, and then more, and more, and more (you get where I'm going). Each new fact uses the previous facts and definitions to show that it too must also be true. This process of building up facts endows mathematics with enormous accountability. By the very nature of our exploration, we can always be confident that the facts we discover are in fact true, no matter how far-fetched or complicated they appear. 

Let's start with one of the most simple definitions - the set of integers. We'll define the set of integers $\mathbb{Z}$ to be the counting numbers $1, 2, 3, ...$ along with $0$, and the negative numbers ${-1}, {-2}, ...$. We can represent the integers in set notation: \\
\\
\centerline{$\mathbb{Z} = \{ ..., {-3}, {-2}, {-1}, 0, 1, 2, 3, ... \}$}. \\
\\
and we'll define an integer to be any number within the set of integers.

The attentive reader may have noticed that in the introduction proof, I justified that the sum of $j$ and $k$ was an integer because both $j$ and $k$ are integers. We'll accept this statement as fact without proving it to be true. Although simply accepting something as fact goes against the very foundation that theoretical mathematics is built on, there are times when we cannot escape having to do so. When we accept a statement as fact without proving it, we call it an axiom. An axiom does not need to be proven; instead, the truth of an axiom is self-evident (think of an axiom like existence - you know you exist, but it's very difficult to quantify a justification for that belief). 

From our axioms and definitions we then derive lemmas and theorems. A lemma is simply a statement that has been proven to be true, such as: "The sum of two even integers is even." Theorems are no different than lemmas in their design; however, they're generally attributed to more important or complex statements. You can think of lemmas as the building blocks to theorems - it is often the case a lemma is proven for the express purpose of being used within the proof of a theorem. 

That's all there is to math theory. We start with some self evident truths - our axioms. We then define an 'object' or two and explore all the truths held within it, spitting out theorems and lemmas along the way. 

\section{The Sum of Two Even Integers is Even}
Before we can take a closer look at this proof, we'll start with definition of what it means to be even: \\

Definition \\
An integer, $n$, is even if $n = 2*j$ for some integer $j$. \\

Now that we have our definition we can begin to tackle this proof. But to begin handling this proof, we first need to determine what exactly it means to show that the sum of two integers is indeed even. Not only that, our argument has to be structured in a way such that regardless of the two even integers chosen, we clearly demonstrate that their sum is even. Now if you're wondering how you could possibly show that it's true for every pair of even integers, or why the proof in the introduction does fulfill this requirement, don't worry - you are not alone! The solution to our problem is actually quite simple, all we have to do is look back to what an even integer is defined to be. If you're wondering how such an abstract definition can help us, again, don't worry, I'm about to explain. Our problem is that we need to show for any two even integers, their sum is even. Well to do that, we need two even integers (obviously!), but we don't want to pick two specific integers. Instead, we need a description of two even integers that could apply to any specific pair. Here we go...\\
\\
Assume I have two even integers $n$ and $m$. \\
\\
Will, really? Yes, really! We need a description that fits any two even integers, and that certainly fits the description. If you're wondering how this statement could possibly be helpful just look back to your definition of an even integer, and you'll find that the statement above contains a bit more information than it lets on at first: \\
\\
Since $n$ is even, there exists some integer $j$ such that $n = 2*j$. \\
Similarly, Since $m$ is also even, there exists some integer $k$ such that $m = 2*k$. \\
\\
The specific values of $j$ and $k$ aren't relevant, all that's important is that they exist, and that we can represent our even integers $n$ and $m$ using them. Let's circle back now to the real problem, how do we show that the sum of $n$ and $m$ is even as well. To do this, let's give the sum a name - let's call it $s$. More formally:\\ 
\\
let $s = n + m$.\\
\\
Once again, we do not care about the specific value of $s$, we only care that it is the sum of $n$ and $m$. Now that we have $s$, we must ask ourselves a very important question - what does it mean to show that $s$ is even. Once again, I refer you to the definition of what it means to be even - $s$ is even if there exists some integer $p$ such that $s = 2*p$. Let's take a recap of all the things we know:\\
\\
- We have two even integers $n = 2*j$ and $m = 2*k$, for some integers $j$ and $k$.\\
- We defined the sum of $n$ and $m$ as the integer $s = n + m$. \\
\\
So what next? Well, let's try substituting $2*j$ for $n$ and $2*k$ for $m$ in our representation of $s$: \\
\\
$s = n + m = 2*j + m = 2*j + 2*k$.\\
\\
Why can we do this? Since $n$ is equal to $2*j$ we can use them interchangeably - they refer to the same exact value. You can think of $2*j$ as a nickname for $n$ - while they may look different, they actually refer to the same thing. Now we all should know what's coming next (after all, I told you only a few paragraphs ago), we are going to factor out the $2$:\\
\\
$s = 2*j + 2*k = 2*(j + k)$. \\
\\
It was no accident that I choose the sum of two integers is an integer as our example axiom; I needed it for what comes next:\\
\\
Since both $j$ and $k$ are integers, their sum $(j + k) = p$ is also an integer.\\
\\
What's the value of $p$? Once again, it's not important, all we need to know is that $p$ is an integer. Finally, we can put it all together by substituting $p$ for $j + k$:\\
\\
$s = 2*(j + k) = 2*p$. Therefor, since $p$ is an integer, and $s$ is equal to $2*p$, $s$ is even by definition. Furthermore, $s$ is not only even, it's also the sum of $n$ and $m$ - which were represented in such a way that our argument could apply to any pair of even integers! \\
\\
And that's all there is to it - we've now proven our statement that the sum of two even integers is even. If you're still a little unsure why the above proves our statement, don't worry - they'll be a few more proofs before the end of this article. If you are uneasy, try re-reading the proof and writing out each step along the way. As you write down each step, ask yourself: why does this argument apply to any integer(s); why am I able to deduce this information from the previous step(s); why is this particular statement true; what am I accomplishing with this step? 

\end{document}